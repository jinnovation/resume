% !TEX TS-program = lualatex

\documentclass[alternative,10pt,compact]{yaac-another-awesome-cv}

\name{Jonathan}{Jin}
\tagline{Machine Learning Infrastructure Engineer}
\socialinfo{
  \website{https://jonathanj.in}{jonathanj.in}
  \email{me@jonathanj.in}
  \linkedin{jinnovation}
  \github{jinnovation}
}

\renewcommand\user[2]{\color{accentcolor}{\LARGE\textbf{#1 #2}}\color{Black}}

\renewcommand\resumetitle[1]{
  %% \ifundef{\@alternative}{
  %%    \par{
  %%    	 \bigskip\center{\Large \color{accentcolor}\textbf{#1}\color{Black}}\par
  %%    }
  %%    \bigskip
  %% }{
    \large{#1}
  %% }
}

\setleftcolumnlength{1.5cm}
\setlength{\rightcolumnlength}{\dimexpr(\rightcolumnlength-0.5cm)\relax}

\newcommand\experiencewithblurb[8]{
  #1    & \textbf{\accentcolor{#2}} \textsc{#3} \hfill #4   \\*
  #5    & \textit{#6} \\*
        & \begin{minipage}[t]{\rightcolumnlength}
            #7
          \end{minipage}									\\*
        & \footnotesize{\foreach \n in {#8}{\cvtag{\n}}} 	\\
}

\newcommand\sectionHeader[1]{\section{\texorpdfstring{\color{accentcolor}\textsc{#1}}{#1}}}

\newcommand\accentcolor[1]{\color{accentcolor}#1\color{Black}}

\begin{document}

%% cls sets French as the default babel language; we undo that here.
\selectlanguage{english}

\makecvheader

%% \makecvfooter
%%     {\textsc{\today}}
%%     {\textsc{Jonathan Jin}}
%%     {}

\sectionHeader{Select Experience}
\begin{experiences}

\experiencewithblurb
    {Present}
    {Spotify}
    {Senior Machine Learning Engineer}
    {New York}
    {03/2021}
    {
      Member of ML Platform. Working on
      \link{https://twimlai.com/conf/twimlcon/2022/session/how-spotify-is-navigating-an-evolving-ml-landscape-with-hendrix-platform/}{Hendrix},
      Spotify's centralized ML platform. My focuses revolve around: ML
      governance; cloud-native infra; and SDK development for MLOps.
    }
    {
      \begin{itemize}

        \item Spearheaded the development of the Hendrix Registry as part of
          company-wide AI/ML governance initiative, delivering a Backstage-based
          \link{https://ai.meta.com/blog/system-cards-a-new-resource-for-understanding-how-ai-systems-work/}{AI
            system} and \link{https://arxiv.org/abs/1810.03993}{model card}
          solution, resulting in over 100 models registered by 40 teams
          company-wide in under a quarter;

        \item Driving cross-functional collaboration between ML Platform and
          Data Platform on data- and ML-native lineage solutions to empower
          responsible AI development, manage costs, and codify org-wide best
          practices and ML development standards;

        \item Led multi-quarter promotion of ML Platform's flagship pipeline
          orchestration product to general availability; collaborated with
          product and engineering stakeholders across peer teams within ML
          Platform as well as ``power user'' teams -- other applied ML teams at
          Spotify -- to: define multi-quarter engineering roadmap; provide
          technical and project leadership/direction. Contributed to an increase
          in ML Platform adoption of 50\% and increase in user satisfaction of
          10\%.

        \item Providing ongoing mentorship and support to three junior IC team
          members, i.e. half of the team's IC makeup; holding bi-weekly
          one-on-ones with each to: provide technical and career mentorship;
          help identify potential future ownership areas and unblock existing
          ones; and address ongoing concerns.

      \item Implemented formalized SLO tracking for multi-cluster, pipeline
        execution infrastructure; used Terraform to canonicalize SLO definitions
        for monitoring and violation reporting in Google Cloud Monitoring;
        implemented custom Kubernetes listener to implement nuanced and
        domain-specific SLIs.

      \item Bootstrapping early-stage development of centralized, managed Ray
        infrastructure based on KubeRay for high-performance ML prototyping and
        experimentation; collaborating with product and engineering leadership
        to concretize our platform's long-term Ray strategy;
      \end{itemize}
    }
    {Ray, Kubernetes, Kubernetes Operators, Go, GCP, Backstage, Terraform, Helm, TensorFlow, TFX, Kubeflow, Prometheus, gRPC}

\emptySeparator

\experiencewithblurb
    {01/2021}
    {NVIDIA}
    {Senior Systems Software Engineer, AI Infrastructure}
    {New York}
    {12/2019}
    {Member of AI Infrastructure. Contributor to
      \link{https://blogs.nvidia.com/blog/2018/09/13/how-maglev-speeds-autonomous-vehicles-to-superhuman-levels-of-safety/}{MagLev},
      NVIDIA’s AI infrastructure for autonomous vehicle development. Also
      contributed to Modulus, the deep learning SDK for autonomous vehicle
      R\&D.}
    {
      \begin{itemize}
      \item Initiated development of solution for ``hybrid data/model
        parallelism'' using a Ray-based parameter server design and Horovod to
        enable horizontally-scalable multi-task training;
      \item Co-delivered a Kubernetes-based scheduling mechanism to enable
        priority access to cluster resources for select use cases, e.g. prep for
        upcoming external demos, via virtual ``resource shares'';
      \end{itemize}
    }
    {Ray, Horovod, TensorFlow, Kubernetes, Helm, gRPC, Bazel, SwiftStack}

\emptySeparator

\experiencewithblurb
    {12/2019}
    {Twitter}
    {Machine Learning Software Engineer}
    {New York}
    {08/2018}
    {Member of \link{http://cortex.twitter.com}{Cortex}, Twitter's central ML
      platform organization. Worked on: workflow orchestration; experiment
      management/iteration; and overall ML engineering productivity.}
    {
      \begin{itemize}
      \item
        Spearheaded initial integration of
        \link{http://tensorflow.org/tfx/}{TensorFlow Extended (TFX)} with
        \link{https://blog.twitter.com/engineering/en_us/topics/insights/2018/ml-workflows.html}{legacy
          Airflow-based orchestration platform} to increase agility of
        workflow development, iterative execution/experimentation, etc.
      \item
        Enabled distributed training of TensorFlow models in Apache Mesos from
        an Airflow pipeline via
        \link{https://blog.twitter.com/engineering/en_us/topics/insights/2018/twittertensorflow.html}{Deepbird},
        Twitter’s TensorFlow-based model training/evaluating/serving framework
      \end{itemize}
    }
    {Apache Airflow, Apache Aurora, TensorFlow}

\end{experiences}

\sectionHeader{Speaking}

\begin{scholarship}

  \scholarshipentry
      {2022}
      {
        \textbf{\link{https://twimlai.com/conf/twimlcon/2022/}{TWIMLcon AI Platforms 2022}},
        \link{https://twimlai.com/conf/twimlcon/2022/session/how-spotify-is-navigating-an-evolving-ml-landscape-with-hendrix-platform/}
        {``How Spotify is Navigating an Evolving ML Landscape with Hendrix Platform''}
      }
  \scholarshipentry
      {2022}
      {
        \textbf{\link{https://mlconf.com/agenda/mlconf-2022-SF/}{MLconf}},
        \link{https://mlconf.com/sessions/empowering-traceable-and-auditable-ml-in-production-at-spotify-with-hendrix/}
        {``Empowering Traceable and Auditable ML in Production at Spotify with Hendrix''}
      }
  \scholarshipentry
      {2021}
      {
        \textbf{\link{https://events.linuxfoundation.org/kubecon-cloudnativecon-north-america}{KubeCon + CloudNativeCon}},
        \link{https://www.youtube.com/watch?v=KUyEuY5ZSqI}
        {``Scaling Kubeflow for Multi-tenancy at Spotify''}
      }

\end{scholarship}

\sectionHeader{Skills}

\begin{keywords}
  \keywordsentry{Programming Languages}
  {
    Python,
    Go,
    Bash,
    C++,
    Java
  }
  \keywordsentry{Machine Learning}
  {
    Ray,
    Kubeflow,
    TensorFlow,
    TensorFlow Extended (TFX),
  }
  \keywordsentry{Distributed Systems}
  {
    Kubernetes,
    gRPC,
    Docker
  }
  \keywordsentry{Infrastructure Tooling}
  {
    Bazel,
    Prometheus,
    Grafana,
    M3,
    Cassandra,
    Apache Airflow
  }

  \keywordsentry{Cloud Infrastructure}
  {
    Google Cloud Platform (GCP),
    Terraform
  }
\end{keywords}

\sectionHeader{Education}

\begin{scholarship}
  \scholarshipentry{2015}{\textbf{University of Chicago}, B.S. Computer Science, B.A. Economics}
\end{scholarship}

\end{document}
