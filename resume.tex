% !TEX TS-program = lualatex

\documentclass[alternative,10pt,compact]{yaac-another-awesome-cv}

\name{Jonathan}{Jin}
\tagline{Machine Learning Infrastructure Engineer}
\socialinfo{
  \website{https://jonathanj.in}{jonathanj.in}
  \email{me@jonathanj.in}
  \linkedin{jinnovation}
  \github{jinnovation}
}

\renewcommand\user[2]{\color{accentcolor}{\LARGE\textbf{#1 #2}}\color{Black}}

\renewcommand\resumetitle[1]{
  %% \ifundef{\@alternative}{
  %%    \par{
  %%    	 \bigskip\center{\Large \color{accentcolor}\textbf{#1}\color{Black}}\par
  %%    }
  %%    \bigskip
  %% }{
    \large{#1}
  %% }
}

\setleftcolumnlength{1.5cm}
\setlength{\rightcolumnlength}{\dimexpr(\rightcolumnlength-0.5cm)\relax}

\newcommand\experiencewithblurb[8]{
  #1    & \textbf{\accentcolor{#2}} \textsc{#3} \hfill #4   \\*
  #5    & \textit{#6} \\*
        & \begin{minipage}[t]{\rightcolumnlength}
            #7
          \end{minipage}									\\*
        & \footnotesize{\foreach \n in {#8}{\cvtag{\n}}} 	\\
}

\newcommand\sectionHeader[1]{\section{\texorpdfstring{\color{accentcolor}\textsc{#1}}{#1}}}

\newcommand\accentcolor[1]{\color{accentcolor}#1\color{Black}}

\begin{document}

%% cls sets French as the default babel language; we undo that here.
\selectlanguage{english}

\makecvheader

%% \makecvfooter
%%     {\textsc{\today}}
%%     {\textsc{Jonathan Jin}}
%%     {}

\sectionHeader{Select Experience}
\begin{experiences}

\experiencewithblurb
    {Present}
    {Spotify}
    {Senior Machine Learning Software Engineer}
    {New York}
    {03/2021}
    {
      Member of ML Platform. Working on: multi-cluster, ML-centric cloud
      infrastructure with Kubernetes and
      \link{http://kubeflow.org/}{Kubeflow}; and ML engineering SDKs, based
      around \link{http://tensorflow.org/tfx/}{TFX}, for pipeline
      orchestration, feature engineering, data processing, model analysis +
      validation, model deployment, etc.
    }
    {
      \begin{itemize}
      \item Spearheading design and development of a unified user-facing
        configuration plane for platform's products; authored and drove
        consensus of architecture and design; guided multi-quarter development
        roadmap and coordinated cross-team collaboration;

      \item Designed formal, opinionated ontology of MLOps semantics to enable
        auditability of ML usage in production and form the basis of Spotify's
        Response AI initiatives, e.g. model cards;

      \item Drove team's primary product (managed Kubernetes cluster
        infrastructure with companion SDK) to general availability; collaborated
        with engineering and product leadership to identify and establish
        baseline product excellence, e.g. API design philosophy, documentation
        tooling, explicit reliability guarantees (SLOs), etc.;

      \item Implemented formalized SLO tracking for multi-cluster, pipeline
        execution infrastructure; used Terraform to canonicalize SLO definitions
        for monitoring and violation reporting in Google Cloud Monitoring;
        implemented custom Kubernetes listener to implement nuanced and
        domain-specific SLIs.

      \item Onboarded four new junior engineers in less than a year; held
        biweekly one-on-ones with each to address concerns, collaborate to
        identify growth opportunities, and provide technical and career
        coaching.
      \end{itemize}
    }
    {Kubernetes, Ray, GCP, Terraform, TensorFlow, TFX, Kubeflow, Prometheus, gRPC}

\emptySeparator

\experiencewithblurb
    {01/2021}
    {NVIDIA}
    {Senior Systems Software Engineer, AI Infrastructure}
    {New York}
    {12/2019}
    {Member of AI Infrastructure. Contributor to
      \link{https://blogs.nvidia.com/blog/2018/09/13/how-maglev-speeds-autonomous-vehicles-to-superhuman-levels-of-safety/}{MagLev},
      NVIDIA’s AI infrastructure for autonomous vehicle development. Also
      contributed to Modulus, the deep learning SDK for autonomous vehicle
      R\&D.}
    {
      \begin{itemize}
      \item Initiated development of solution for ``hybrid data/model
        parallelism'' using a Ray-based parameter server design and Horovod to
        enable horizontally-scalable multi-task training;
      \item Co-delivered a Kubernetes-based scheduling mechanism to enable
        priority access to cluster resources for select use cases, e.g. prep for
        upcoming external demos, via virtual ``resource shares'';
      \end{itemize}
    }
    {Kubernetes, TensorFlow, Horovod, Ray, gRPC, Bazel, SwiftStack}

\emptySeparator

\experiencewithblurb
    {12/2019}
    {Twitter}
    {Machine Learning Software Engineer}
    {New York}
    {08/2018}
    {Member of \link{http://cortex.twitter.com}{Cortex}, Twitter's central ML
      platform organization. Worked on: workflow orchestration; experiment
      management/iteration; and overall ML engineering productivity.}
    {
      \begin{itemize}
      \item
        Spearheaded initial integration of
        \link{http://tensorflow.org/tfx/}{TensorFlow Extended (TFX)} with
        \link{https://blog.twitter.com/engineering/en_us/topics/insights/2018/ml-workflows.html}{legacy
          Airflow-based orchestration platform} to increase agility of
        workflow development, iterative execution/experimentation, etc.
      \item
        Enabled distributed training of TensorFlow models in Apache Mesos from
        an Airflow pipeline via
        \link{https://blog.twitter.com/engineering/en_us/topics/insights/2018/twittertensorflow.html}{Deepbird},
        Twitter’s TensorFlow-based model training/evaluating/serving framework
      \end{itemize}
    }
    {Apache Airflow, Apache Aurora, TensorFlow}

\end{experiences}

\sectionHeader{Skills}

\begin{keywords}
  \keywordsentry{Programming Languages}
  {
    Python,
    Go,
    Bash,
    C++,
    Java
  }
  \keywordsentry{Machine Learning}
  {
    Kubeflow,
    TensorFlow Extended (TFX),
    TensorFlow,
    Ray
  }
  \keywordsentry{Distributed Systems}
  {
    Kubernetes,
    gRPC,
    Docker
  }
  \keywordsentry{Infrastructure Tooling}
  {
    Bazel,
    Prometheus,
    Grafana,
    M3,
    Cassandra,
    Apache Airflow
  }

  \keywordsentry{Cloud Infrastructure}
  {
    Google Cloud Platform (GCP),
    Terraform
  }
\end{keywords}

\sectionHeader{Education}

\begin{scholarship}
  \scholarshipentry{2015}{\textbf{University of Chicago}, B.S. Computer Science, B.A. Economics}
\end{scholarship}

\end{document}
